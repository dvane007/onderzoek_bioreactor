\documentclass[a4paper, 11pt, leqno]{article}

\author{Diederik van Engelenburg (4365402)}
\title{Huiswerk 1, TW1070}
\date{\today}


%pakketten.
\usepackage[dutch]{babel}

%commands
\newcommand{ \dd }{\mathrm{d}}

\begin{document}

\section*{Bioreactor}

We beschouwen een eenvoudig model voor een bioreactor, beschreven door de volgende gekoppelde differentiaalvergelijkingen
\begin{eqnarray*}
	\frac{\dd X}{\dd t} &=& \alpha_1 \frac{S}{1 + S} X - X \\
	\frac{\dd S}{\dd t} &=& - \frac{1}{1 + S}X - S + \alpha_2
\end{eqnarray*}
met $X$ en $S$ functies van (de tijd) $t \geq 0$, en $\alpha_1, \alpha_2$ positieve constanten. De functie $X$ staat voor de dichtheid van de bacterie\"en in de reactor en de functie $S$ beschrijft de concentratie van de voeding voor bacteri\"en.

\begin{itemize}

\item Zoek informatie op internet over de bovengegeven vergelijkingen. 

\item Geef een interpretatie van de termen in de differentiaalvergelijkingen. 

\item Bedenk wat je onder evenwicht zou kunnen verstaan en bepaal wat hier de evenwichtswaarden zijn.

\item Bepaal geschikte waarden voor de constanten $\alpha_1, \alpha_2$ en beginwaarden $X(0), S(0)$, en los de vergelijkingen op met de methode van Euler. 

\item Geef de gevonden oplossingen weer in afzonderlijke plots. Geef de oplossingen ook weer in het $X-S$ vlak, het zogenaamde fasevlak. 

\item Verklaar het gedrag van de gevonden oplossingen ook door in de buurt van de evenwichtswaarden te kijken naar de linearisatie. 

\item Ga op zoek naar een zinvolle uitbreiding van het bovengegeven model. Bijvoorbeeld, ga op zoek naar een meer gedetailleerde beschrijving van de bioreactor en analyseer deze. 

\end{itemize}

\end{document}