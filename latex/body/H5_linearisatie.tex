% !TEX root = ../main.tex

\chapter{Linearisatie.}
\label{Linearisatie}

In het vorige hoofdstuk hebben we gezien dat de voedsel- en bacterieconcentratie altijd naar een evenwicht toe convergeren. In dit hoofdstuk gaan we deze evenwichten nader beschouwen door middel van linearisatie.

Linearisatie is een methode die differentiaalvergelijkingen benadert door middel van eerste orde differentiaalvergelijkingen, zo eerste orde differentiaalvergelijkingen gelijk zijn aan de originele vergelijkingen in het evenwicht daarvan. Het mooie van eerste orde differentiaalvergelijkingen is dat ze altijd een exacte oplossing hebben. Met deze eigenschap kunnen we de aard van de evenwichten van de originele differentiaalvergelijkingen bestuderen.

We beschouwen de vergelijkingen zoals in hoofdstuk \ref{Introductie_vergelijkingen} en laten hier $F(X,S) = \frac{\dd X}{\dd t}$ en $G(X,S) = \frac{\dd S}{\dd t}$. Nu is de eerste stap van de linearisatie het berekenen van de parti\"ele afgeleiden van $F$ en $G$.
\begin{align*}
F_X(X,S) &= \frac{\dd}{\dd X}(\alpha_1 \frac{S}{1 + S} X - X) = \alpha_1\frac{S}{1 + S} - 1\\
F_S(X,S) &= \frac{\dd}{\dd S}(\alpha_1 \frac{S}{1 + S} X - X) = \alpha_1\frac{X}{(1+S)^2}\\
G_X(X,S) &= \frac{\dd}{\dd X}(-\frac{S}{1 + S}X - S + \alpha_2) = -\frac{S}{1 + S}\\
G_S(X,S) &= \frac{\dd}{\dd S}(-\frac{S}{1 + S}X - S + \alpha_2) = -1-\frac{X}{(1+s)^2}
\end{align*}

Laat nu $(X_0, S_0)$ een evenwichtspunt zijn van het originele stelsel differentiaalvergelijkingen, dan
\[\frac{\dd}{\dd t}\left( \begin{array}{c}
X-X_0 \\
S-S_0 \end{array} \right)
= \left( \begin{array}{cc}
F_X(X_0,S_0) & F_S(X_0,S_0) \\
G_X(X_0,S_0) & G_S(X_0,S_0) \end{array} \right)
\left( \begin{array}{c}
X-X_0 \\
S-S_0 \end{array} \right).\]

Uit de eigenwaarden van de Jacobi-matrix $\left[ \begin{array}{cc}
F_X(X_0,S_0) & F_S(X_0,S_0) \\
G_X(X_0,S_0) & G_S(X_0,S_0) \end{array} \right]$ kunnen we nu het type evenwicht bepalen.

We beschouwen eerst het evenwicht $(X_0, S_0) = (0, \alpha_2)$. Dan volgt de volgende matrix:
\[J = \left[ \begin{array}{cc}
F_X(X_0,S_0) & F_S(X_0,S_0) \\
G_X(X_0,S_0) & G_S(X_0,S_0) \end{array} \right] =
\left[ \begin{array}{cc}
\frac{\alpha_1\alpha_2}{1+\alpha_2}-1 & 0 \\
\frac{\alpha_2}{1+\alpha_2} & -1 \end{array} \right]\]
De eigenwaarden van deze matrix zijn $-1$ en $\frac{\alpha_1\alpha_2}{1+\alpha_2}-1$. Als $\alpha_2 < \frac{1}{1-\alpha_1}$ dan zijn beide eigenwaarden kleiner dan 0 en is het punt $(0, \alpha_2)$ een stabiel evenwicht. Zo niet, dan is het een zadelpunt.

Nu beschouwen we het andere evenwicht $(X_0, S_0) = (\alpha_1\alpha_2-\frac{1}{\alpha_1-1}-1,\frac{1}{\alpha_1-1})$. Dan volgt de volgende matrix:
\[J = \left[ \begin{array}{cc}
F_X(X_0,S_0) & F_S(X_0,S_0) \\
G_X(X_0,S_0) & G_S(X_0,S_0) \end{array} \right] =
\left[ \begin{array}{cc}
0 & (\alpha_1\alpha_2-\alpha_2-1)(\alpha_1-1) \\
-\frac{1}{\alpha_1} & -\frac{\alpha_2(\alpha_1-1)^2+1)}{\alpha_1} \end{array} \right]
\]
De eigenwaarden van deze matrix zijn $-1$ en $\frac{(1-\alpha_1)(\alpha_1\alpha_2-\alpha_2-1)}{\alpha_1}$. Als $\alpha_2<\frac{1}{\alpha_1-1}$, dan zijn beide eigenwaarden kleiner dan 0 en is het punt $(\alpha_1\alpha_2-\frac{1}{\alpha_1-1}-1,\frac{1}{\alpha_1-1})$ een stabiel evenwicht. Zo niet, dan is het een zadelpunt.