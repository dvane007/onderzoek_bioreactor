% !TEX root = ../main.tex

\chapter{Uitbreidingen.}
\label{Uitbreidingen}

Zoals we zagen in hoofdstuk 3, groeit in ons model de bacterieconcentratie steeds sneller naarmate de toevoer van voedsel toeneemt. Een bioreactor heeft echter beperkte ruimte en dus is de bacterieconcentratie naar boven begrensd. Een begrenzing van de bacterieconcentratie zou dus een zinvolle uitbreiding zijn op het model. Aangezien de inhoud van een reactor een constante waarde heeft, betekent dit enkel dat $S + X$ nooit groter mag zijn dan de inhoud. Hiertoe hoeven we alleen geschikte beginwaarden te kiezen. 

Daarnaast is het algemeen bekend dat bacteri\"en niet alleen leven op voedsel, maar ook op andere factoren, zoals zuurstof en water. Deze kunnen natuurlijk worden verpakt in de voedselconcentratie, maar ik denk dat het realistischer is om een van deze, of beide, weer te geven in een aparte vergelijking. 

Bovendien zou er gekeken kunnen worden naar factoren zoals licht en warmte, die de groei van bacteri\"en sterk be\"invloed. Deze kunnen echter ook verpakt worden in de constante $\alpha_1$. 

In het algemeen, natuurlijk afhankelijk van waarvoor de bioreactor gebruikt zal gaan worden, lijkt het mij handig om een kosten-analyse te maken, door zowel het voedsel als de bacteri\"en een kostprijs te geven (en een winstkenmerk). In zo'n geval kunnen we daadwerkelijk opzoek gaan naar het meest voordelige evenwicht in een bioreactor met inhoud $I$. 

