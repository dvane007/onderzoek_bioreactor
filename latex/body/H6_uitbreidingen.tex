% !TEX root = ../main.tex

\chapter{Uitbreidingen.}
\label{Uitbreidingen}

Omdat een bioreactor een beperkte inhoud heeft, zou dit een goede uitbreiding van het model zijn. Zoals we zagen in hoofdstuk 3, groeit in ons model de bacterieconcentratie steeds sneller naarmate de toevoer van voedsel toeneemt. Door het model uit te breiden met een beperkte inhoud, kan dit niet meer. Zo wordt het model realistischer. Noemen we nu de inhoud van de reactor $I$, zo dat $X + S \leq I$, dan kunnen we hieruit de mogelijke waarde $\alpha_2$ afleiden. Voor het evenwicht $(X,S) = (0,\alpha_2)$ zien we nu dat moet gelden dat $\alpha_2 < I$. Voor het evenwicht $(X,S) = (\alpha_1\alpha_2-\frac{1}{\alpha_1-1}-1, \frac{1}{\alpha_1-1})$ zien we nu dat $\alpha_1\alpha_2-\frac{1}{\alpha_1-1}-1 + \frac{1}{\alpha_1-1} = \alpha_1\alpha_2 - 1 \leq I$ ofwel
\begin{equation}
\alpha_2 \leq \frac{I + 1}{\alpha_1}. \label{eq:boundary}
\end{equation}

Daarnaast is het algemeen bekend dat bacteri\"en niet alleen leven op voedsel, maar ook op andere factoren, zoals zuurstof en water. Deze kunnen natuurlijk worden verpakt in de voedselconcentratie, maar het is realistischer om een van deze, of beide, weer te geven in een aparte vergelijking. 

Bovendien zou er gekeken kunnen worden naar omgevingsfactoren zoals licht en warmte, die de groei van bacteri\"en sterk be\"invloed. Deze worden in het huidige model verpakt in de constante $\alpha_1$, maar kunnen ook los worden genomen voor een meer realistisch model. 

Een andere uitbreiding van het model is een kosten-analyse. Als we aannemen dat de kostprijs van bacteri\"en en voedsel constant is, kunnen we hiermee op zoek gaan naar het meest voordelige evenwicht in een bioreactor met een eindige inhoud $I$. Deze uitbreiding zullen we verder uitwerken.

Laten we de totale kosten van de reactor $K(t)$ noemen. Ten eerste is deze afhankelijk van de opstartkosten $K(0)$. Deze bestaan uit de bouw- en materiaalkosten $r$ van de reactor uit de kosten van het voedsel en de bacteri\"en die op $t=0$ aan de reactor worden toegevoegd. Noemen we de prijs van een hoeveelheid voedsel $a$ en de prijs van een hoeveelheid bacteri\"en $b$, dan worden de totale opstartkosten $K(0) = r + aS(0) + bX(0)$.

Naast de opstartkosten hebben we nog een aantal variabele kosten. Ten eerste het voedsel. Per tijdseenheid wordt, volgens de vergelijkingen in hoofdstuk \ref{Inleiding} $\alpha_2$ voedsel toegevoegd. Bovendien stroomt er $S(t)$ voedsel uit de reactor. Stel dat we van dit uitstromende voedsel een deel $0\leq c\leq1$ terug kunnen winnen zodat we het opnieuw kunnen gebruiken. dan worden de totale kosten van het voedsel per tijd $a(\alpha_2-cS(t))$. 

Ten tweede de bacteri\"en. Per tijdseenheid stroomt er $X(t)$ bacteri\"en de reactor uit. Stel dat we een deel $0\leq d\leq1$ van deze uitstromende bacteri\"en verkopen, dan worden de totale kosten van de bacteri\"en per tijd $b(-dX(t))$.

Ten slotte hebben we nog te maken met kosten om de reactor draaiend te houden. We nemen aan dat deze kosten constant zijn, en noemen deze $k$.

Als we al de bovengenoemde kosten combineren krijgen we de volgende differentiaalvergelijking:
\begin{equation}
\frac{\dd K}{\dd t} = a(\alpha_2 - cS(t)) - bdX(t) + k. \label{eq:cost}
\end{equation}

Als we nu aannemen dat de reactor het meest van de tijd in evenwicht is, kunnen we bij gegeven $\alpha_1$ een $\alpha_2$ bepalen zo dat de kosten per tijd zo laag mogelijk zijn.

De kosten per tijd geven we weer in een functie $f$ afhankelijk van $\alpha_2$. Hiertoe vullen we de evenwichtsvoorwaarden in in vergelijking \ref{eq:cost} en stellen we $f(\alpha_2) = \frac{\dd K}{\dd t}$. We krijgen nu:
\begin{align*}
f(\alpha_2)
&=a(\alpha_2 - \frac{c}{\alpha_1-1}) - bd(\alpha_1\alpha_2-\frac{1}{\alpha_1-1}-1) + k\\
&=\alpha_2(a-\alpha_1bd) + \frac{bd-ac}{\alpha_1-1}+bd+k.
\end{align*}
Dit is een lijn.

Om de minimale kosten te vinden moeten we nu het minimum van deze functie vinden voor $\alpha_2\in[0, \frac{I + 1}{\alpha_1}]$, want \ref{eq:boundary}. Als $a > \alpha_1bd$, dan is $f$ een stijgende lijn en zijn de kosten minimaal als $\alpha_2 = 0$. Het is nu zaak voedsel ergens anders goedkoper in te kopen. Als $a = \alpha_1bd$, dan is $f$ constant en zijn de kosten niet afhankelijk van $\alpha_2$. Als $a < \alpha_1bd$, dan is $f$ een dalende lijn. In dit geval kunnen we een lucratieve business opzetten. De kosten per tijdseenheid zijn nu het kleinst als $\alpha_2 = \frac{I+1}{\alpha_1}$. Nu geldt:
\begin{align*}
\frac{\dd K}{\dd t}
&=\frac{I+1}{\alpha_1}(a-\alpha_1bd) + \frac{bd-ac}{\alpha_1-1}+bd+k\\
&=a\frac{I+1}{\alpha_1} + \frac{bd-ac}{\alpha_1-1} - Ibd + k
\end{align*}

De kosten per tijd hangen nu alleen nog maar af van $\alpha_1$ en de inhoud $I$. Omdat $\alpha_1$ gegeven is voor de bacteriesoort, kunnen we alleen de inhoud $I$ nog vari\"eren. Laat nu $g(I) = \frac{\dd K}{\dd t}$ dan krijgen we de volgende functie:
\begin{align*}
g(\alpha_1)
&=a\frac{I+1}{\alpha_1} + \frac{bd-ac}{\alpha_1-1} - Ibd + k\\
&=I\left(\frac{a}{\alpha_1}-bd\right) + \frac{a}{\alpha_1} + \frac{bd-ac}{\alpha_1-1} + k
\end{align*}
Dit is een lijn.

Om de minimale kosten te vinden moeten we nu het minimum van deze functie vinden voor $I\in[0,\inf)$.

Als $\alpha_1<\frac{a}{bd}$, dan is $g$ een stijgende lijn en zijn de kosten minimaal als $I = 0$. Om de kosten te drukken, moet er of goedkoper voedsel worden ingekocht, of moeten er meer bacteri\"en worden verkocht voor een hogere prijs.

Als $\alpha_1=\frac{a}{bd}$, dan is $g$ constant en hangen de kosten niet af van $I$. Als $\alpha_1>\frac{a}{bd}$, dan is $g$ een dalende lijn en kunnen we door $I$ groot te kiezen, uitkomen op negatieve kosten, ofwel winst! Hiervoor moet gelden dat $I\left(\frac{a}{\alpha_1}-bd\right) + \frac{a}{\alpha_1} + \frac{bd-ac}{\alpha_1-1} + k < 0$, ofwel $I > \frac{\frac{a}{\alpha_1} + \frac{bd-ac}{\alpha_1-1} + k}{bd-\frac{a}{\alpha_1}}$.








