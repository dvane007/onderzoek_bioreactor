% !TEX root = ../main.tex

\chapter{Uitbreidingen.}
\label{Uitbreidingen}

Zoals we zagen in hoofdstuk 3, groeit in ons model de bacterieconcentratie steeds sneller naarmate de toevoer van voedsel toeneemt. Een bioreactor heeft echter beperkte ruimte en dus is de bacterieconcentratie naar boven begrensd. Een begrenzing van de bacterieconcentratie zou dus een zinvolle uitbreiding zijn op het model. Aangezien de inhoud van een reactor een constante waarde heeft, betekent dit enkel dat $S + X$ nooit groter mag zijn dan de inhoud. Hiertoe hoeven we alleen geschikte beginwaarden te kiezen. 

Daarnaast is het algemeen bekend dat bacteri\"en niet alleen leven op voedsel, maar ook op andere factoren, zoals zuurstof en water. Deze kunnen natuurlijk worden verpakt in de voedselconcentratie, maar ik denk dat het realistischer is om een van deze, of beide, weer te geven in een aparte vergelijking. 

Bovendien zou er gekeken kunnen worden naar factoren zoals licht en warmte, die de groei van bacteri\"en sterk be\"invloed. Deze kunnen echter ook verpakt worden in de constante $\alpha_1$. 

In het algemeen, natuurlijk afhankelijk van waarvoor de bioreactor gebruikt zal gaan worden, lijkt het mij handig om een kosten-analyse te maken, door zowel het voedsel als de bacteri\"en een kostprijs te geven (en een winstkenmerk). In zo'n geval kunnen we daadwerkelijk opzoek gaan naar het meest voordelige evenwicht in een bioreactor met inhoud $I$. 

Hier zullen we iets dieper op ingaan. Laten we de vergelijking voor de kosten van de reactor $K(t)$ noemen. Afhankelijk van de precieze vraag rekenen we de kosten van de reactor zelf mee. Deze kosten noemen we $r_0$. Daarnaast hebben we op $t = 0$ ook nog andere kosten, die afhangt van $S(0)$ en $X(0)$. Laten we zeggen dat, als $S = 1$, dan zijn de kosten voor het voedsel $a$. Voor bacteri\"en nemen we dat als $X = 1$, dan zijn de kosten $b$. Er volgt dus dat $K(0) = r_0 + aS(0) + bX(0)$. 

Dan zijn er nog een aantal variabele kosten. Ten eerste het voedsel. In het tijdsinterval $t$ wordt er, volgens de vergelijkingen in hoofdstuk \ref{Inleiding} $\alpha_2$ voedsel toegevoegd. Bovendien stroomt er $S(t)$ voedsel uit de reactor. Er zijn nu een twee mogelijkheden. (i) het voedsel dat uit de reactor stroomt kan niet meer worden teruggewonnen. In dat geval zien we $\frac{\dd K}{\dd t} = a \alpha_2$. (ii) het voedsel dat uit de reactor stroomt kan wel worden teruggewonnen. In dat geval zien we $\frac{\dd K}{\dd t} = a (\alpha_2 - S(t))$. 

Ten tweede de bacteri\"en. Ook hiervoor zijn een aantal mogelijkheden. De eerste mogelijkheid is dat de bacteri\"en die uit de reactor stromen verkocht worden. Dan volgt dat $\frac{\dd K}{\dd t} = -bX(t)$. Als ze niet verkocht worden, maar bijvoorbeeld gebruikt worden voor onderzoek, hebben ze geen invloed op de kosten van de reactor. 

Als laatste hebben we nog te maken met gemiddelde onderhoudskosten en de kosten om de reactor \"uberhaupt te laten draaien. We nemen aan dat deze kosten constant zijn, en noemen deze $k$. Al met al volgt er dus zoiets als:

\begin{align*}
	K(0) = r_0 + aS(0) + bX(0) \\
	\frac{\dd K}{\dd t} = a (\alpha_2 - S(t)) - bX(t),
\end{align*}
waarbij $X$ en $S$ hetzelfde verband hebben als in hoofdstuk \ref{Inleiding}.
