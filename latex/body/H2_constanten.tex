% !TEX root = ../main.tex

%
%document: beschrijft constatenten alpha_1, alpha_2. Hierin ook methode van Euler verder toelichten en toepassen op een
%aantal waarden van alpha_1, alpha_2 en beredeneren wat geschikte X(0) en S(0) zijn (in termen van alpha_1 en alpha_2. 
%
% Bronnen: http://www.chem.ucla.edu/dept/Faculty/merchant/pdf/microbial.pdf, http://www.ncbi.nlm.nih.gov/pmc/articles/
% PMC1449012/, http://www.ncbi.nlm.nih.gov/pmc/articles/PMC372817/?page=15.

%begin hoodstuk. 
\chapter{Bepaling van constanten.}
\label{Bepaling van constanten}

De vergelijking voor de groei van bacteri\"en zoals eerder beschreven hangt natuurlijk af van het type bacterie waarmee we rekenen en van een aantal factoren naast het voedsel, zoals licht, warmte en dergelijken. We spreken bij bacteri\"en ook wel van een 'specific growth rate', gemeten per uur. 

\section{$\alpha_1$, specifieke-groeifactor}
Algemeen bekend is dat bacteri\"en zich vermenigvuldigen doormiddel van celdeling. Daarbij ontstaan uit \'e\'en bacterie twee nieuwe, identieke bacteri\"en. In dit model rekenen we met gewicht per inhoudsmaat, maar die zijn min of meer evenredig met de hoeveelheid bacteri\"en in de reactor. Hierdoor weten we dus dat er, uitgaande van het gegeven dat er een beginhoeveelheid $X(0)$ bacteri\"en is, $X(t) = X(0) \cdot 2^{t_d \cdot t}$, waarbij $t_d$ een constante delingsfactor is, die verschilt per type bacterie.We kunnen dit natuurlijk herschrijven tot $X(t) = X(0) \cdot e^{\alpha_1}$, waarbij $\alpha_1$ weer een constante is. Dit is ook de vergelijking zoals we die beschrijven in hoofdstuk \ref{Introductie_vergelijkingen}. Deze $\alpha_1$ word de specifieke, maximale groeifactor genoemd. 

In het vervolg zullen enkel nog gegevens gebruiken van de bacteriesoort E. Coli (vanwege veelvuldig onderzoek naar deze bacteriesoort). De precieze keuze van de groeifactor $\alpha_1$ is dan ook niet van enorm belang voor ons onderzoek, aangezien we de meeste berekeningen doen zonder $\alpha_1$ vast te kiezen. We laten hier een aantal waarden voor de groeifactor zien, zodat er een realistischer context voor het onderzoek wordt geschapen. 
\\
\\
Afhankelijk van een aantal factoren, zoals licht, warmte etc. is onderzoek gedaan naar verschillende groei-waarden voor E. Coli \footnote{namen hier.. (2006), Stad: , American Society for Microbiology}. We zien dan dat de waarden sterk verschillen. In ons onderzoek hebben we niet zoveel aan waarden voor $\alpha_1\leq1$, aangezien dan de hoeveelheid bacteri\"en constant daalt. Dus moet er gebruik worden gemaakt van een omgeving, zodat we bijvoorbeeld $\alpha_1 = 2.4$ krijgen. We zien dat er waarden bestaan tussen $1$ en $2.4$ \footnote{N.B. volgt uit het onderzoek van de A.S.M. dat er ook maximale-groeiconstanten bestaan die kleiner zijn dan $1$, zoals $0.7$ bij bepaalde licht-waarden. Deze waarden resulteren bij ons onderzoek in een krimp van bacteri\"en, dus het evenwicht $(X(t), S(t)) = (0, \alpha_2t + S(0)$) }. De precieze waarde zal dus afhangen van een omstandigheden waarvoor gekozen wordt.


\section{$\alpha_2,$ toevoer van voedsel.}
In ons model stroomt met een constante snelheid $\alpha_2$ de bioreactor in. We willen in ieder geval dat de concentratie bacteri\"en constant blijft, ofwel $\alpha_1\alpha_2-\frac{1}{\alpha_1-1}-1 = 0$ en dus $\alpha_2 = \frac{\frac{1}{\alpha_1-1}+1}{\alpha_1}=\frac{1}{\alpha_1-1}$.

De optimale instroom snelheid is de snelheid waarbij de verhouding tussen de uitstroom van bacteri\"en en de instroom van voedsel het grootst is. De uitstroom van bacteri\"en per tijdseenheid is gelijk aan $X(t)$. We gaan er vanuit dat de bioreactor lang blijft draaien en dus het grootste deel van zijn tijd in evenwicht zal zijn. In dat geval geldt $X = \alpha_1\alpha_2-\frac{1}{\alpha_1-1}-1$, waarbij $\alpha_1$ de maximale groei is behorende bij de bacterie soort. De toevoer van voedsel $\alpha_2$ is variabel en dus is de verhouding tussen de uitstroom van bacteri\"en en de instroom van voedsel een functie $r$ die afhangt van $\alpha_2$:
\[r(\alpha_2) = \frac{\alpha_1\alpha_2-\frac{1}{\alpha_1-1}-1}{\alpha_2} = \alpha_1-\frac{1}{\alpha_2(\alpha_1-1)}-\frac{1}{\alpha_2}.\]
Dit is een stijgende functie. Hoe meer voedsel aan de reactor toegevoegd wordt, hoe hoger de concentratie bacteri\"en bij evenwicht zal zijn en hoe hoger het rendement zal zijn. De concentratie bacteri\"en heeft natuurlijk een maximum, dat we verder zullen bespreken in hoofdstuk (??). Het is in ieder geval duidelijk dat de voedsel instroom een positief verband heeft met het rendement en met de bacterie concentratie, wanneer de reactor in evenwicht is. Voor nu houden instroom van voedsel op $\alpha_2 = \frac{1}{\alpha_1-1} + x$ met $x\in[0, 5]$.

\section{$X(0), S(0)$ beginwaarde van de bacteri\"e- en voedselconcentratie.}
Het doel van de bioreactor is zoveel mogelijk bacteri\"en te kweken uit zo weinig mogelijk. We moeten $X(0)$ dus zo laag mogelijk kiezen. We kunnen niet beginnen met $X(0) = 0$, want dan zouden er geen bacteri\"en zijn en ook niet komen. Dus we kiezen een getal dicht bij nul, 0.1.
De beginhoeveelheid voedsel moet zo zijn, dat het aantal bacteri\"en in ieder geval niet afneemt, ofwel $\frac{\dd X}{\dd t} = 0$. Uit vergelijking (2.5) volgt dan dat $S(0) = \frac{1}{\alpha_1 - 1}$.

Nu we goede beginwaarden hebben bepaald, kunnen we gaan kijken naar een aantal verlopen van de bacterie - en voedselconcentratie.
