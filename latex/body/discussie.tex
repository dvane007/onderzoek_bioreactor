% !TEX root = ../main.tex

\chapter{Discussie.}
\label{Discussie}
Als we de resultaten van ons onderzoek op een rijtje zetten, kunnen we stellen dat lang niet alles van belang is. Zo zijn evenwichten waarvoor geldt dat $X$ naar 0 toegaat niet erg handig om te gebruiken in de praktijk. Natuurlijk is het wel van belang te weten wanneer dit gebeurt, zodat deze situaties vermeden kunnen worden. 

Daarnaast is het van belang om te kijken naar hoe realistisch de resultaten zijn. Zoals ook al werd aangestipt in hoofdstuk \ref{Uitbreidingen} (Uitbreidingen), is er een maximale inhoud voor de bioreactor. Dus zijn alleen die resultaten van belang waarbij $X + S \leq I$, met $I$ de inhoud van de bioreactor. Aan de andere kant kun je dit ook omdraaien. Als je $x$ bacteri\"en wilt hebben, dan geeft onze methode een redelijke benadering van het evenwicht en kun je snel uitrekenen hoe groot de reactor moet zijn.

Wat ons onderzoek vooral zou kunnen verbeteren is echter een onderzoek in de praktijk. Uitgaande van onderzoek zoals in bron 1, zou er een opstelling gemaakt kunnen worden, die de bioreactor nabouwt en onze evenwichten in de praktijk probeert toe te passen. Uit dit onderzoek zou kunnen volgen of en waar we al dan niet factoren over het hoofd hebben gezien die in de werkelijkheid wel van groot belang blijken te zijn. 

Daarbij komt ook kijken dat het allicht handig is om wat dieper in te gaan op de constante factoren, zoals licht en warmte. Bovendien zouden deze factoren ook variabel kunnen zijn (als dit in kosten scheelt bijvoorbeeld). Als laatste factor lijkt mij het ook realistisch om te kijken naar de overige inhoud van de reactor (waarschijnlijk gewoon lucht) en wat voor invloed dit heeft op de bacteriegroei. 

%einde