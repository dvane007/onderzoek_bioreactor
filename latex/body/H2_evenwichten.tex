% !TEX root = ../main.tex


%onderdeel van chapter 1, eerste analyse vergelijkingen.
\chapter{Introductie in de vergelijkingen.} 			%betere titel bedeken!
\label{Introductie_vergelijkingen}

%sectie: vergelijkingen analyseren.
\section{Interpretatie van de vergelijkingen.}
De vergelijkingen zoals beschreven in hoofdstuk 1, de inleiding, komen niet uit de lucht vallen. We zullen de interpretatie van deze vergelijkingen hier analyseren.

\subsection*{Bacterie-concentratie}
We beginnen met de analyse van de bacterie-groei. In eerste instantie beschouwen we hiertoe vat bacteri\"en in optima-forma, oftewel: er is geen enkele beperkende factor met betrekking tot de groei van de kolonie en we bekijken een omgeving waarin geen bacteri\"en worden weggenomen. We weten dat de groei van de bacteri\"en dan slechts afhangt van het aantal aanwezige bacteri\"en en de soort bacterie. Ofwel $\frac{\dd X}{\dd t} = \alpha_1X$ met $\alpha_1>0$ is de maximale groei specifiek voor deze bacterie. Het oplossen van deze differentiaalvergelijking geeft de vergelijking
\begin{equation*}
	X(t) = X(0) e^{\alpha_1t},		
\end{equation*}
waarbij $X(t)$ de concentratie bacteri\"en is t.o.v. de tijd $t$.

Een van de meest voor de hand liggende beperkende factoren is de voedsel-concentratie $S$. We noemen de functie van de beperkende factor ten opzichte van de voedselconcentratie $b(S)$, zodat de hele vergelijking nu $\frac{\dd X}{\dd t} = \alpha_1b(S)X$ wordt. Deze functie moet aan een aantal voorwaarden voldoen. Als er geen voedsel is kunnen de bacteri\"en niet groeien, ofwel $b(0) = 0$. Als er heel veel voedsel is, worden de bacteri\"en niet beperkt in hun groei en is deze groei dus gelijk aan de groei zonder beperkingen, ofwel $lim_{S\rightarrow\infty}b(S)=1$.
Een functie die aan deze voorwaarden voldoet is de functie $b(S) = \frac{S}{1+S}$. We kiezen deze functie als functie van de beperkende factor, zo dat de totale vergelijking de volgende wordt:
\begin{equation*}
	\frac{\dd X} {\dd t} = \alpha_1 \frac{S}{1 + S} X
\end{equation*}
%Dit doen we zodat hoe hoger het voedsel, hoe dichter we tegen de maximale groei ($\alpha_1$) aan gaan zitten. Met dit verband houden we dus rekening met drie eisen: hoe hoger de voedselconcentratie, hoe hoger de bacterie groei, de bacterie-groei is nooit hoger dan de maximum of optimale groei en als de voedselconcentratie nul is, is de toename van bacteri\"en ook nul. Daarmee zitten we bij een redelijk goed verband. De 1 onder de deler in de factor $\frac{S}{1+S}$ zal enkel afhangen van de eenheden waarin gemeten wordt. 

Het vat bacteri\"en, of de bioreactor, is echter niet alleen een plaats waar bacteri\"en groeien. We willen namelijk ook constant een deel van de bacteri\"en wegnemen. Dit weghalen gebeurt door middel van een uitstroom. Bij deze bioreactor stroomt er elke tijdseenheid $t$, $X(t)$ bacteri\"en weg. Als we dit toevoegen aan de vergelijking krijgen we de totale vergelijking:
\begin{equation}
	\frac{\dd X}{\dd t} = \alpha_1 \frac{S}{1+S} X - X 		\label{eq:bc1}
\end{equation}
Dit is precies de vergelijking zoals beschreven in hoofdstuk 1. 

\subsection*{Voedselconcentratie}
Vervolgens analyseren we de vergelijking van de voedselconcentratie $S$ in de bioreactor. Wederom beschouwen we eerst een reactor zonder in- of uitstroom. De voedselconcentratie hangt dan enkel af van de concentratie bacteri\"en aanwezig in de reactor en de eetlust van deze bacteri\"en. De vergelijking is nu $\tfrac{\dd S}{\dd t} = -e(S)X$, met $e(S)$ is de eetlust van de bacteri\"en ten opzichte van de concentratie voedsel. $e(S)$ moet weer voldoen aan een aantal voorwaarden. Als $S = 0$, dan kan de voedselconcentratie niet afnemen, dus $e(0)=0$. Ook willen we dat naarmate de voedselconcentratie groter wordt, de eetlust toeneemt en dat de eetlust nooit groter wordt dan 1. Een functie die hieraan voldoet is $e(S) = \frac{S}{1+S}$. De totale vergelijking wordt nu
\[ 
	\frac{\dd S}{\dd t}=-\frac{S}{1+S}X.
\]
We zien nu, dat de voedselconcentratie alleen maar af kan nemen. Dit willen we niet, want, dan zou op een gegeven moment het voedsel op zijn. Om dit te compenseren moeten we dus voedsel toevoeren aan de reactor. Als we per tijdseenheid $t$, $\alpha_2$ voedsel toevoeren, krijgen we nu de vergelijking
\[ 
	\frac{\dd S}{\dd t}=-\frac{S}{1+S}X + \alpha_2.
\]
Als we de bioreactor nu zo uitbreiden dat er ook nog uitstroom is, dan weten we dat (aangezien de eenheden gelijk zijn) moet gelden dat er een afname van voedsel is, die een recht-evenredig verband heeft met de voedselconcentratie op tijdstip $t$. Dus is er een afname van $a S(t)$ op ieder tijdstip. Afhankelijk van de eenheden en de grootte van de uitstroom wordt de constante $a$ gekozen. In dit geval kiezen we $a = 1$, vanwege onder andere de hoeveelheid bacteri\"en die uitstroomt op tijdstip $t$. Bovendien gebruiken we voor zowel de bacterieconcentratie als de voedselconcentratie dezelfde eenheden. Dus volgt er dat de vergelijking voor de verandering in voedselconcentratie er als volgt uitziet:
\begin{equation}
	\frac{\dd S}{\dd t} = -\frac{S}{1 + S} X - S + \alpha_2.
\end{equation}
Wat precies de vergelijking is zoals beschreven in hoofdstuk \ref{Inleiding}.

%sectie: evenwichten: ----------------------------------------------------------------------------------------------------------------
\section{Berekening evenwicht.}
We beschouwen de differentiaalvergelijkingen zoals beschreven in hoofdstuk \ref{Inleiding}:  
\begin{align}
	\frac{\dd X}{\dd t} &= \alpha_1 \frac{S}{1 + S} X - X 			\label{eq:be1}	\\
	\frac{\dd S}{\dd t} &= - \frac{S}{1 + S}X - S + \alpha_2 		\label{eq:be2}
\end{align}
Deze reactor is in evenwicht als zowel de concentratie bacteri\"en als de concentratie voedsel constant is. Ofwel, als $\frac{\dd X}{\dd t} = 0$ en $\frac{\dd S}{\dd t} = 0$.

We bekijken eerst vergelijking (\ref{eq:be1}).

Uit $\tfrac{\dd X}{\dd t} = 0$, volgt dat of $X = 0$ of $X \neq 0$. Als $X \not = 0$ volgt:
\begin{align*}
	\alpha_1 \frac{S}{1 + S} X - X = 0 &\iff \\
	X = \alpha_1 \frac{S}{1 + S}X &\iff \\
	\alpha_1 = \frac{1 + S}{S} &\iff \\
	\alpha_1S-S = 1 &\iff \\
	S(\alpha_1-1) = 1
\end{align*}
Ofwel
\begin{equation}
S=\frac{1}{\alpha_1-1}												\label{eq:be3}
\end{equation}

Deze vergelijking nemen we mee naar het evenwicht van het voedsel. Immers, als we te maken hebben met een evenwicht, moet zowel $\tfrac{\dd X}{\dd t} = 0$ als $\tfrac{\dd S}{ \dd t } = 0$ gelden. We weten dat de bacterie-concentratie in evenwicht is als $X(t) = 0$. In dat geval volgt uit $\tfrac{\dd S}{\dd t} = 0$, dat $-S+\alpha_2=0$, ofwel $S = \alpha_2$.

Uit $X\neq0$ en vergelijkingen (\ref{eq:be2}) en (\ref{eq:be3}) vinden we dan dat, als $\tfrac{\dd S}{\dd t} = 0$,
\begin{align*}
	- \frac{S}{1 + S}X - S + \alpha_2 = 0 		&\iff \\
	\frac{S}{1 + S}X = \alpha_2 - S 			&\iff \\
	X = \frac{\alpha_2 + \alpha_2 S - S - S^2}{S} &\iff \\
	X = \alpha_2 \left( \frac{1}{S} + 1 \right) - 1 - S
\end{align*}
Substitueren van $S = \frac{1}{\alpha_1-1}$ geeft
\begin{equation}
	X = \alpha_2 \left( \frac{\alpha_1 - 1}{1} + 1 \right) - 1 - \frac{1}{\alpha_1 - 1} = \alpha_1\alpha_2 - \frac{1}{\alpha_1 - 1} - 1
\end{equation}
De evenwichten van het stelsel zijn dus
\[
	(X,S) = (0, \alpha_2)
\]
en
\[
	(X,S) = \left( \alpha_1\alpha_2 - \frac{1}{\alpha_1 - 1} - 1, \frac{1}{\alpha_1-1} \right)
\]

 %einde sectie: Berekening Evenwichten.--------------------------------------------------------------------------------------------------

