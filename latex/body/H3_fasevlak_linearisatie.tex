% !TEX root = ../main.tex

\chapter{Fasevlak en Linearisatie.}

\section{Het fasevlak.}

We bekijken eerst een aantal fase-vlakken voor verschillende constanten $\alpha_1$ en $\alpha_2$, respectievelijk de maximale bacterie-groei en de toename van voedsel. Hiertoe nemen we $\alpha_1$ zoals we die vinden in hoofdstuk \ref{Bepaling van constanten}. Zoals we al eerder zagen, in hoofdstuk \ref{Introductie_vergelijkingen}, leveren alle waarden $ \alpha_1 \leq 1$ wel een evenwicht op, maar altijd zo dat $X(t) \to 0$ als $t$ groter wordt. Daarom kiezen we waarden $1 < \alpha_1  < 2.4$ (zie hiertoe weer hoofdstuk \ref{Bepaling van constanten}). Voor het berekenen van het fase-vlak gebruiken we de methode van Euler\footnote{Zie bijlage ?? voor de precieze implementatie van deze methode.}.

%fasevlak invoegen

%fasevlak invoegen.

%fasevlak invoegen.

Uit de bovenstaande fasevlakken zien we drie duidelijke evenwichten ontstaan. 