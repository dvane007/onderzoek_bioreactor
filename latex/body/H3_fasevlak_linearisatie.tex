% !TEX root = ../main.tex

\chapter{Fasevlak en Linearisatie.}
\label{Fasevlak en Linearisatie}

\section{Het fasevlak.}

We bekijken eerst een aantal fase-vlakken voor verschillende constanten $\alpha_1$ en $\alpha_2$, respectievelijk de maximale bacterie-groei en de toename van voedsel. Hiertoe nemen we $\alpha_1$ zoals we die vinden in hoofdstuk \ref{Bepaling van constanten}. Zoals we al eerder zagen, in hoofdstuk \ref{Introductie_vergelijkingen}, leveren alle waarden $ \alpha_1 \leq 1$ wel een evenwicht op, maar altijd zo dat $X(t) \to 0$ als $t$ groter wordt. Daarom kiezen we waarden $1 < \alpha_1  < 2.4$ (zie hiertoe weer hoofdstuk \ref{Bepaling van constanten}). Voor het berekenen van het fase-vlak gebruiken we de methode van Euler\footnote{Zie bijlage ?? voor de precieze implementatie van deze methode.}.

%fasevlak invoegen

%fasevlak invoegen.

%fasevlak invoegen.

Uit de bovenstaande fasevlakken zien we drie duidelijke evenwichten ontstaan. 

%meer tekst hier.

\section{Linearisatie.}

We beschouwen de vergelijkingen voor bacterie-groei en voedseltoename zoals in hoofdstuk \ref{Introductie_vergelijkingen}. Om de evenwichten zoals we die vinden bij het berekenen van de evenwichten nader te beschouwen, maken we een linearisatie van de vergelijkingen. Hiertoe berekenen we eerst de parti\"ele afgeleiden van de twee vergelijkingen. 

We weten dat er twee type evenwichten zijn; namelijk met $\frac{\dd S}{\dd t} = \alpha_2$ en $\frac{\dd X}{\dd t} = 0$, en $\frac{\dd S}{\dd t} = \frac{1}{\alpha_1 - 1} $ en $\frac{\dd X}{\dd t} = \frac{\alpha_1(\alpha_1 \alpha_2 - \alpha_2 - 1)}{\alpha_1 - 1}$, zoals we zien in sectie ??. Voor de eerste weten we dat er een stabiel evenwicht ontstaat, met de jacoba-matrix:
\begin{equation*}
	\begin{pmatrix}
		
	\end{pmatrix}
	
\end{equation*}

